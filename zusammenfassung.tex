\section{Zusammenfassung und Diskussion}
\label{zusammenfassung}
\thispagestyle{plain}

In der vorliegenden Arbeit wurde ein Szenarium f"ur die koordinierte
Entwicklung von Okulardominanz-- und Orientierungskarten entwickelt.  Wie
in Kapitel~\ref{main} gezeigt, bietet die aus der Sicht der Nichtlinearen
Dynamik plausible Annahme dynamischer Umordnung und Interaktion der
Okulardominanz-- und Orientierungspr"aferenzkarten eine einfache und
einheitliche Erkl"arung f"ur die unterschiedliche Erscheinung dieser
Strukturen im visuellen Cortex von Affen und Katzen.  Das streifige
Erscheinungsbild der OD--Karte aus V1 des Affen kann durch dynamische
Umordnung des Musters nach seiner Entstehung, in Abwesenheit
orientierungsspeziefischer Zellen erkl"art werden.  Die perlige Struktur
der OD--Karte in A17 der Katze wird durch die Wechselwirkung des Musters
der Okulardominanz mit der bereits etablierten Orientierungskarte
erkl"art. Diese f"uhrt dazu, da"s die Grenzlinien der entstehenden
OD--Dom"anen h"aufig geschlossene Kurven um Pinwheels bilden und verhindert
die Umordnung der OD--Karte in ein anisotropes Muster mit globaler
Vorzugsrichtung.

\begin{quote}
In diesem Bild wird das Layout einer neuronalen Karte also
entscheidend durch den \emph{Zeitpunkt} ihrer Entstehung bestimmt.  Dieser
spezielle Zeitpunkt bestimmt auch die Wellenl"ange des Musters: Die
Vorhersage des sequentiellen Bifurkationsszenariums ist, da"s in allen Spezies
die Karte mit der \emph{gr"o"seren} Wellenl"ange \emph{fr"uher} entsteht.
\end{quote}

In V1 des Affen sind nach der Geburt sowohl orientierungsselektive als auch
okulardominante Zellen vorhanden~\cite{blakemore:1978}, daher ist keine sichere
Aussage "uber die zeitliche Abfolge ihrer Spezialisierung m"oglich.
\citeasnoun{rakic:1981} beschreibt allerdings, da"s die
Okulardominanzsegregation ca.~3~Wochen vor der Geburt einsetzt. Aus
A17 der Katze ist bekannt, da"s hier die Zellen in einem fr"uhen
Stadium der Entwicklung bereits orientierungsselektiv sind
\citeaffixed{movshon:1981,fregnac:1984}{siehe}, w"ahrend 
die Okulardominanzkarte erst im Laufe der 3.--7. Woche nach der Geburt entsteht
\cite{shatz:1978,levay:1979}. Dies stimmt mit der Vorhersage des
sequentiellen Bifurkationsszenariums "uberein.

\citeasnoun{jones:1991} haben eine alternative Erkl"arung des
unterschiedlichen Erscheinungsbildes von Okulardominanzkarten in Katzen und
Affen vorgeschlagen. Die Autoren argumentieren, da"s der hohe Grad
r"aumlicher Koh"arenz der OD--Karte im Affen Anisotropien der retinotopen
Abbildung vom Gesichtsfeld in den Cortex reflektiert. Das isotrope OD--Muster
in der Katze ist demnach Ausdruck einer isotropen retino--corticalen
Abbildung.  Die Autoren benutzen ein nicht weiter biologisch motiviertes
Optimierungsverfahren zur numerischen Evaluation ihrer
Hypothese,\footnote{\citeasnoun{bauer95b} konnte jedoch zeigen, da"s sich
neuronale Merkmalskarten unter diesen Annahmen "ahnlich Verhalten.} und
behandeln die Okulardominanzstruktur unabh"angig von anderen, im visuellen
Cortex vorhandenen Reizrepr"asentationen.  Die in Abschn.~\ref{90grad}
vorgestellte Wechselwirkung zwischen den Mustern der Okulardominanz und der
Orientierungspr"aferenz l"a"st die isolierte Untersuchung der
Okulardominanz jedoch fragw"urdig erscheinen.
\setcounter{footnote}{1}

Es ist wichtig anzumerken, da"s die in Kapitel~\ref{main} vorgestellten
Simulationsergebnisse keineswegs von der speziellen Wahl der
Dynamik~\eqref{endyn} abh"angen, vielmehr jedoch von den folgenden,
allgemeinen Eigenschaften:
 
\begin{itemize}
\item die Gr"o"se eines typischen, durch einen Stimulus hervorgerufenen
Erregungsgebietes im Cortex bestimmt die Wellenl"ange der entstehenden
kolumn"aren Struktur
\item diese Gr"o"se nimmt mit der Zeit ab
\item kolumn"are Strukturen entstehen nur unterhalb einer kritischen
Gr"o"se des Erregungsgebietes. Oberhalb dieser kritischen
Kooperationsreichweite bleibt die homogene L"osung stabil.
\end{itemize}

Jedes Modell, das diese allgemeinen Eigenschaften mit der
Dynamik~\eqref{endyn} gemeinsam hat, ist geeignet die Entstehung
kolumn"arer Muster unterschiedlicher Wellenl"ange anhand des sequentiellen
Bifurkationsszenariums zu beschreiben. Das hier verwendete
elastische Netz ist also nur ein Vertreter aus einer viel umfangreicheren
Klasse von Modellen. Die Verwendung des elastischen Netzes rechtfertigt
sich dabei vor allem durch den leichten analytischen Zugang zu den
Musterbildungsmechanismen dieses Modells, sowie durch den vertretbaren,
numerischen Aufwand bei der Implementierung der Dynamik auf Computern.

Die konsequente Interpretation der aktivit"atsabh"angigen
Selbstorganisation neuronaler Karten als dynamischer Prozess dient auch der
einheitlichen Erkl"arung eines anderen Speziesunterschieds.
Orientierungskarten aus V1 des Affen und A17 strabismischer und
normalsichtiger Katzen unterscheiden sich in der der Dichte ihrer
Punktdefekte: Wird zum Speziesvergleich die Dichte der Pinwheels $\rho
[1/mm^2]$ einer Orientierungskarte mit der charakteristischen Wellenl"ange
$\Lambda_{\text{OP}}$ dieser Karte skaliert, so ergibt sich im Affen ein
Wert von $\hat{\rho}=\rho \Lambda_{\text{OP}}^2\approx 3.75$ ($\hat{\rho}$
mi"st dann die relative H"aufigkeit der Pinwheels in einem K"astchen der
Seitenl"ange $\Lambda_{\text{OP}}$). In der Katze ist diese Gr"o"se immer
merklich kleiner als $3$, nach vorliegenden, z.T. unver"offentlichten
Messungen betr"agt der Wert $\hat{\rho}\approx 2.4$ f"ur normalsichtige und
$\hat{\rho}\approx 3.0$ f"ur strabismische Katzen.

\citeasnoun{wolf:1996} konnten zeigen, da"s die skalierte Dichte in allen
Spezies anf"anglich gr"o"ser als $\pi$ sein mu"s. In Tieren, in denen der
adulte Wert von $\hat{\rho}$ deutlich unter $\pi$ liegt mu"s dieser
geringere Wert folglich durch Annihilation von Pinwheels entgegengesetzter
Chiralit"at in Folge eines Umordnungsprozesses der Karte w"ahrend der
Entwicklung erreicht worden sein. Das Auftreten unterschiedlicher
skalierter Pinwheeldichten $\hat{\rho}$ in verschiedenen Spezies kann unter
Ber"ucksichtigung der Wechselwirkung zwischen den Mustern der
Okulardominanz und der Orientierungspr"aferenz durch die beobachtete
Korrelation der Pinwheeldichte mit dem Grad der Okulardominanzsegregation
erkl"art werden.

Im Folgenden versuchen wir die Frage zu beantworten warum die kritische
Kooperationsreichweite der Okulardominanz im Affen, wie im sequentiellen
Bifurkationsszenarium gefordert, gr"o"ser sein sollte als die der
Orientierungspr"aferenz (und aus welchen Gr"unden dieses Verh"altnis in der
Katze dann umgekehrt ist).

Die Tendenz der Neurone, sich auf Reize aus einem Auge zu spezialisieren
wird in biologischen Systemen durch die unvollst"andige Korrelation der
Aktivit"at in den beiden Augen bestimmt: Es ist daher anzunehmen, da"s
diese Tendenz der Neurone mit abnehmender Korrelation zwischen beiden Augen
zunimmt.  Dies wird auch durch Beobachtungen an strabismischen Katzen
belegt: Durch Induzierung eines k"unstlichen Schielwinkels sind die
Aktivit"aten beider Augen v"ollig dekorreliert und f"uhren zur
vollst"andigen Segregation der Afferenzen aus beiden Augen
(vgl. Abb.~\ref{okuhist}, Mitte).

Der kritische Wert der Kooperationsreichweite $\sigma^\ast_{\text{OD}}$ ist
im Modell ein Ma"s f"ur die Tendenz der Neurone, sich auf Signale aus einem
der beiden Augen zu spezialisieren. Je gr"o"ser $\sigma^\ast_{\text{OD}}$,
desto gr"o"ser mu"s die Kooperationsreichweite der Neurone sein um die
Segregation zu verhindern. Es ist daher anzunehmen, da"s dieser kritische
Wert von den Korrelationen beider Augen bestimmt wird: Wachsende
Korrelationen sollten zu einer Verkleinerung des kritischen Wertes f"uhren.
Dies ist in der Tat in detaillierten Modellen der Okulardominanzsegregation
der Fall \cite{scherf:1994}. 

Im Affen beginnt die Okulardominanzsegregation bereits vor der Geburt
\cite{rakic:1981} und erfolgt daher unter dem Einflu"s
notwendigerweise dekorrelierter Spontanaktivit"at der beiden Retinae.  In
Katzen hingegen vollzieht sich die Entwicklung des visuellen Cortex erst
nach der Geburt, und erfolgt weitgehend unter dem Einflu"s korrelierter
Aktivit"at aus beiden Augen nach Augen"offnung.  Mit diesem Zusammenhang
kann auch die in strabismischen Katzen beobachtete Vergr"o"serung der
Okulardominanzwellenl"ange gegen"uber normalsichtigen Katzen erkl"art
werden: Der erh"ohte Wert von $\sigma^\ast_{\text{OD}}$ f"uhrt analog zum
sequentiellen Bifurkationsszenarium zur fr"uheren Entstehung des Musters,
und damit zu einer Vergr"o"serung der Wellenl"ange
\citeaffixed{scherf:1994}{siehe auch}.

Diese Erkl"arung legt eine interessante Verifikation des hier entwickelten
Szenariums nahe: Sollte es durch fr"uhste Dekorrelation der Eingangsaktivit"at
der Augen m"oglich sein, die Okulardominanzsegregation zu induzieren, bevor
sich das Muster der Orientierungspr"aferenzen ausbildet, so m"u"ste sich in
diesen Katzen ein ``affen"ahnliches'' OD--Muster ausbilden.  Chronisches
optisches Ableiten neuronaler Karten w"ahrend des Entwicklungszeitraumes
wird in zuk"unftigen Experimenten am visuellen Cortex einzelner Katzen oder
Affen eine geeignete M"oglichkeit bieten, die in dieser Arbeit
vorgeschlagenen Mechanismen auf ihre G"ultigkeit hin zu "uberpr"ufen.

\nocite{hoffsuemmer95a,hoffsuemmer96}

