\section{Zusammenfassung und Diskussion}
\label{zusammenfassung}
\thispagestyle{plain}

In der vorliegenden Arbeit wurde ein Szenarium für die koordinierte
Entwicklung von Okulardominanz-- und Orientierungskarten entwickelt. Wie
in Kapitel~\ref{main} gezeigt, bietet die aus der Sicht der Nichtlinearen
Dynamik plausible Annahme dynamischer Umordnung und Interaktion der
Okulardominanz-- und Orientierungspräferenzkarten eine einfache und
einheitliche Erklärung für die unterschiedliche Erscheinung dieser
Strukturen im visuellen Cortex von Affen und Katzen. Das streifige
Erscheinungsbild der OD--Karte aus V1 des Affen kann durch dynamische
Umordnung des Musters nach seiner Entstehung, in Abwesenheit
orientierungsspeziefischer Zellen erklärt werden. Die perlige Struktur
der OD--Karte in A17 der Katze wird durch die Wechselwirkung des Musters
der Okulardominanz mit der bereits etablierten Orientierungskarte
erklärt. Diese führt dazu, daß die Grenzlinien der entstehenden
OD--Domänen häufig geschlossene Kurven um Pinwheels bilden und verhindert
die Umordnung der OD--Karte in ein anisotropes Muster mit globaler
Vorzugsrichtung.

\begin{quote}
    In diesem Bild wird das Layout einer neuronalen Karte also
    entscheidend durch den \emph{Zeitpunkt} ihrer Entstehung bestimmt. Dieser
    spezielle Zeitpunkt bestimmt auch die Wellenlänge des Musters: Die
    Vorhersage des sequentiellen Bifurkationsszenariums ist, daß in allen Spezies
    die Karte mit der \emph{größeren} Wellenlänge \emph{früher} entsteht.
\end{quote}

In V1 des Affen sind nach der Geburt sowohl orientierungsselektive als auch
okulardominante Zellen vorhanden~\cite{blakemore:1978}, daher ist keine sichere
Aussage über die zeitliche Abfolge ihrer Spezialisierung möglich.
\textcite{rakic:1981} beschreibt allerdings, daß die
Okulardominanzsegregation ca.~3~Wochen vor der Geburt einsetzt. Aus
A17 der Katze ist bekannt, daß hier die Zellen in einem frühen
Stadium der Entwicklung bereits orientierungsselektiv sind
\parencite[siehe][]{movshon:1981,fregnac:1984}, während
die Okulardominanzkarte erst im Laufe der 3.--7. Woche nach der Geburt entsteht
\cite{shatz:1978,levay:1979}. Dies stimmt mit der Vorhersage des
sequentiellen Bifurkationsszenariums überein.

\textcite{jones:1991} haben eine alternative Erklärung des
unterschiedlichen Erscheinungsbildes von Okulardominanzkarten in Katzen und
Affen vorgeschlagen. Die Autoren argumentieren, daß der hohe Grad
räumlicher Kohärenz der OD--Karte im Affen Anisotropien der retinotopen
Abbildung vom Gesichtsfeld in den Cortex reflektiert. Das isotrope OD--Muster
in der Katze ist demnach Ausdruck einer isotropen retino--corticalen
Abbildung. Die Autoren benutzen ein nicht weiter biologisch motiviertes
Optimierungsverfahren zur numerischen Evaluation ihrer
Hypothese,\footnote{\textcite{bauer95b} konnte jedoch zeigen, daß sich
neuronale Merkmalskarten unter diesen Annahmen ähnlich Verhalten.} und
behandeln die Okulardominanzstruktur unabhängig von anderen, im visuellen
Cortex vorhandenen Reizrepräsentationen. Die in Abschn.~\ref{90grad}
vorgestellte Wechselwirkung zwischen den Mustern der Okulardominanz und der
Orientierungspräferenz läßt die isolierte Untersuchung der
Okulardominanz jedoch fragwürdig erscheinen.
\setcounter{footnote}{1}

Es ist wichtig anzumerken, daß die in Kapitel~\ref{main} vorgestellten
Simulationsergebnisse keineswegs von der speziellen Wahl der
Dynamik~\eqref{endyn} abhängen, vielmehr jedoch von den folgenden,
allgemeinen Eigenschaften:

\begin{itemize}
    \item die Größe eines typischen, durch einen Stimulus hervorgerufenen
    Erregungsgebietes im Cortex bestimmt die Wellenlänge der entstehenden
    kolumnären Struktur
    \item diese Größe nimmt mit der Zeit ab
    \item kolumnäre Strukturen entstehen nur unterhalb einer kritischen
    Größe des Erregungsgebietes. Oberhalb dieser kritischen
    Kooperationsreichweite bleibt die homogene Lösung stabil.
\end{itemize}

Jedes Modell, das diese allgemeinen Eigenschaften mit der
Dynamik~\eqref{endyn} gemeinsam hat, ist geeignet die Entstehung
kolumnärer Muster unterschiedlicher Wellenlänge anhand des sequentiellen
Bifurkationsszenariums zu beschreiben. Das hier verwendete
elastische Netz ist also nur ein Vertreter aus einer viel umfangreicheren
Klasse von Modellen. Die Verwendung des elastischen Netzes rechtfertigt
sich dabei vor allem durch den leichten analytischen Zugang zu den
Musterbildungsmechanismen dieses Modells, sowie durch den vertretbaren,
numerischen Aufwand bei der Implementierung der Dynamik auf Computern.

Die konsequente Interpretation der aktivitätsabhängigen
Selbstorganisation neuronaler Karten als dynamischer Prozess dient auch der
einheitlichen Erklärung eines anderen Speziesunterschieds.
Orientierungskarten aus V1 des Affen und A17 strabismischer und
normalsichtiger Katzen unterscheiden sich in der der Dichte ihrer
Punktdefekte: Wird zum Speziesvergleich die Dichte der Pinwheels $\rho
[1/mm^2]$ einer Orientierungskarte mit der charakteristischen Wellenlänge
$\Lambda_{\text{OP}}$ dieser Karte skaliert, so ergibt sich im Affen ein
Wert von $\hat{\rho}=\rho \Lambda_{\text{OP}}^2\approx 3.75$ ($\hat{\rho}$
mißt dann die relative Häufigkeit der Pinwheels in einem Kästchen der
Seitenlänge $\Lambda_{\text{OP}}$). In der Katze ist diese Größe immer
merklich kleiner als $3$, nach vorliegenden, z.T. unveröffentlichten
Messungen beträgt der Wert $\hat{\rho}\approx 2.4$ für normalsichtige und
$\hat{\rho}\approx 3.0$ für strabismische Katzen.

\textcite{wolf:1996} konnten zeigen, daß die skalierte Dichte in allen
Spezies anfänglich größer als $\pi$ sein muß. In Tieren, in denen der
adulte Wert von $\hat{\rho}$ deutlich unter $\pi$ liegt muß dieser
geringere Wert folglich durch Annihilation von Pinwheels entgegengesetzter
Chiralität in Folge eines Umordnungsprozesses der Karte während der
Entwicklung erreicht worden sein. Das Auftreten unterschiedlicher
skalierter Pinwheeldichten $\hat{\rho}$ in verschiedenen Spezies kann unter
Berücksichtigung der Wechselwirkung zwischen den Mustern der
Okulardominanz und der Orientierungspräferenz durch die beobachtete
Korrelation der Pinwheeldichte mit dem Grad der Okulardominanzsegregation
erklärt werden.

Im Folgenden versuchen wir die Frage zu beantworten warum die kritische
Kooperationsreichweite der Okulardominanz im Affen, wie im sequentiellen
Bifurkationsszenarium gefordert, größer sein sollte als die der
Orientierungspräferenz (und aus welchen Gründen dieses Verhältnis in der
Katze dann umgekehrt ist).

Die Tendenz der Neurone, sich auf Reize aus einem Auge zu spezialisieren
wird in biologischen Systemen durch die unvollständige Korrelation der
Aktivität in den beiden Augen bestimmt: Es ist daher anzunehmen, daß
diese Tendenz der Neurone mit abnehmender Korrelation zwischen beiden Augen
zunimmt. Dies wird auch durch Beobachtungen an strabismischen Katzen
belegt: Durch Induzierung eines künstlichen Schielwinkels sind die
Aktivitäten beider Augen völlig dekorreliert und führen zur
vollständigen Segregation der Afferenzen aus beiden Augen
(vgl. Abb.~\ref{okuhist}, Mitte).

Der kritische Wert der Kooperationsreichweite $\sigma^\ast_{\text{OD}}$ ist
im Modell ein Maß für die Tendenz der Neurone, sich auf Signale aus einem
der beiden Augen zu spezialisieren. Je größer $\sigma^\ast_{\text{OD}}$,
desto größer muß die Kooperationsreichweite der Neurone sein um die
Segregation zu verhindern. Es ist daher anzunehmen, daß dieser kritische
Wert von den Korrelationen beider Augen bestimmt wird: Wachsende
Korrelationen sollten zu einer Verkleinerung des kritischen Wertes führen.
Dies ist in der Tat in detaillierten Modellen der Okulardominanzsegregation
der Fall \cite{scherf:1994}.

Im Affen beginnt die Okulardominanzsegregation bereits vor der Geburt \cite{rakic:1981}
und erfolgt daher unter dem Einfluß notwendigerweise dekorrelierter Spontanaktivität der
beiden Retinae. In aKatzen hingegen vollzieht sich die Entwicklung des visuellen Cortex erst
nach der Geburt, und erfolgt weitgehend unter dem Einfluß korrelierter
Aktivität aus beiden Augen nach Augenöffnung. Mit diesem Zusammenhang
kann auch die in strabismischen Katzen beobachtete Vergrößerung der
Okulardominanzwellenlänge gegenüber normalsichtigen Katzen erklärt
werden: Der erhöhte Wert von $\sigma^\ast_{\text{OD}}$ führt analog zum
sequentiellen Bifurkationsszenarium zur früheren Entstehung des Musters,
und damit zu einer Vergrößerung der Wellenlänge
\parencite[siehe auch][]{scherf:1994}.

Diese Erklärung legt eine interessante Verifikation des hier entwickelten
Szenariums nahe: Sollte es durch frühste Dekorrelation der Eingangsaktivität
der Augen möglich sein, die Okulardominanzsegregation zu induzieren, bevor
sich das Muster der Orientierungspräferenzen ausbildet, so müßte sich in
diesen Katzen ein ``affenähnliches'' OD--Muster ausbilden. Chronisches
optisches Ableiten neuronaler Karten während des Entwicklungszeitraumes
wird in zukünftigen Experimenten am visuellen Cortex einzelner Katzen oder
Affen eine geeignete Möglichkeit bieten, die in dieser Arbeit
vorgeschlagenen Mechanismen auf ihre Gültigkeit hin zu überprüfen. \nocite{hoffsuemmer95a,hoffsuemmer96}

