\section{Einleitung}
\label{einleitung}
\thispagestyle{plain}

\subsection{Physik im Hirn?} 
%... stellt der darwinistischen evolution die sozio-kulturelle evolution zur
%seite, welche dieser teilweise entgegenwirkt.

Verglichen mit anderen Lebewesen auf diesem Planeten nimmt die Spezies
Mensch eine Ausnahmeposition ein.  Diese Sonderstellung wird nicht nur
durch morphologische Unterschiede zu unseren n"achsten Verwandten wie
z.B. aufrechter Gang und umgestaltetes Gebi"s gerechtfertigt. Sie f"allt
dem Menschen vielmehr durch die besondere Auspr"agung seines Gehirns,
insbesondere des Neocortex zu (siehe Abb.~\ref{hirnmasse}).  Das
menschliche Gehirn ist die komplexeste, uns bekannte Struktur im Universum.
Es besteht aus ca. $10^{12}$~Nervenzellen, von denen jede einzelne mit bis
zu tausenden anderen verbunden ist, und ist materielle und funktionale
Grundlage von Wahrnehmen, Empfinden, Erkennen, Lernen, Erinnern, Denken und
Mitteilen; auf seinen Prozessen beruht auch das Bewu"stsein. Gleichwohl ist
das Gehirn sich selbst r"atselhaft.

Seit ungef"ahr 100~Jahren gibt es ernstzunehmende Ans"atze von
Naturwissenschaftlern, dieses R"atsel zu l"osen.  Mittlerweile sind zu der
wissenschaftlichen Neugier klare wirtschaftliche und industrielle
Interessen hinzugekommen. So arbeiten in diesem Jahrzehnt --- vom Kongress
der Vereinigten Staaten zur ``Dekade des Hirns'' ausgerufen --- weltweit
Wissenschaftler aus allen Disziplinen daran, die Funktionsweise des
menschlichen Gehirns zu entschl"usseln. Von den Ergebnissen dieser
Forschungen erhofft man sich nicht nur ein besseres Verst"andnis
neurologischer Krankheitsbilder, wie z.B. der Parkinson'schen Krankheit und
der Multiplen Sklerose. (Dies k"onnte den Schl"ussel zu deren Heilung,
zumindest jedoch Chancen auf bessere Therapien beinhalten.)

Eine besondere Motivation entspringt der Hoffnung, durch die Extraktion
einfacher Regeln und Funktionsprinzipien, die der Informationsverarbeitung
im menschlichen Gehirn zugrunde liegen, die Leistungsf"ahigkeit
k"unstlicher informationsverarbeitender Systeme bei bestimmten
Aufgabenstellungen (wie z.B. Mustererkennung, Sprachverst"andnis,
Spracherkennung, Bewegungskoordination) massiv zu steigern. Auch der sich
durch die ganze Kulturgeschichte ziehende Traum des Menschen von einem
ebenb"urtigen, von Menschenhand geschaffenen Wesen, dem \emph{Humanoiden},
w"are dann in greifbare N"ahe ger"uckt \citeaffixed{brooks:1990}{zum Stand
der Dinge siehe z.B.}.

Auf dem Weg zu einem einheitlichen Verst"andnis von Hirnentwicklung und
\mbox{--funktion} k"onnen auch Physiker einen gewinnbringenden Beitrag leisten.
Die moderne Physik bietet eine Palette von Werkzeugen und Methoden zur
Analyse und Beschreibung komplexer Nicht--Gleichgewichts--Systeme.  Viele
der in der Theoretischen Physik an einfachen Systemen entwickelten Konzepte
k"onnen auf die Hirnforschung "ubertragen werden, und so auf einer
abstrakten Ebene zum qualitativen Verst"andnis beobachteter Ph"anomene
beitragen. Idealerweise werden dabei in der Theorie entwickelte Konzepte
und Modelle im Wechselspiel mit dem Experiment verfeinert. Oft lassen sich
aus solchen Modellen auch Vorhersagen und neue Fragestellungen ableiten.

Diese Erg"anzung der experimentellen Hirnforschung um theoretische
Komponenten hat sich in der Vergangenheit bereits f"ur beide Gebiete als
fruchtbar erwiesen: Durch Arbeiten von \citeasnoun{hopfield:1982}
z.B. wurde die statistische Physik als geeigneter Rahmen f"ur die Analyse
der Ged"achtnisfunktion sowie der Verallgemeinerung beim Erlernen von
Regeln entdeckt.  Die Untersuchung pulsgekoppelter Neurone lieferte neue
Erkenntnisse "uber die nichtlinearen Synchronisationsmechanismen
oszillierender Systeme \citeaffixed{ernst95a}{siehe z.B.}.  In dieser
Arbeit werden Konzepte der Nichtlinearen Dynamik strukturbildender Systeme
verwendet, um Unterschiede in den visuellen Reizrepr"asentationen im Hirn
von Katzen und Affen einheitlich zu erkl"aren.  Langfristig dient die
Ausweitung des Kanons der Physik auf biologische Systeme wie z.B. dem
Gehirn wohl dem Erkenntnisfortschritt in beiden Gebieten, der Physik und
der Biologie \cite{braitenberg:1977}.

\begin{figure}[t]
\begin{center}
\epsfig{file=pics/hirngewicht.eps,width=9cm}
\end{center}
\caption{Man kann die "Uberentwicklung des menschlichen Hirns bereits
deutlich am Verh"altnis zwischen Gehirn-- und K"orpergewicht ablesen. Hier
dargestellt ist dieses Verh"altnis in doppeltlogarithmischer Darstellung
f"ur 476 S"augetierarten und den Menschen
\protect\citeaffixed{martin:1995}{aus}. Das prozentuale Verh"altnis erg"abe
eine Abweichung zugunsten leichtgewichtigerer Arten.  Der Wert f"ur den
Menschen liegt am weitesten oberhalb der Allometriegeraden; unsere Spezies
hat also relativ zum K"orpergewicht ``am meisten'' Hirn.}
\label{hirnmasse}
\end{figure}

\subsection{"Uberblick}

In der vorliegenden Arbeit wird ein Modell zur Beschreibung der
aktivit"atsabh"angigen, koordinierten Entwicklung neuronaler
Reizrepr"asentationen (auch: neuronale Karten) untersucht.  Basierend auf
diesem Modell wird ein Szenarium vorgeschlagen, das die speziesabh"angige
Erscheinung der Reizrepr"asentationen aus dem prim"aren visuellen Cortex
von Katzen und Affen einheitlich erkl"aren kann.  Die Arbeit gliedert sich
wie folgt:

\begin{itemize}
\item Eine Einf"uhrung in die wichtigsten biologischen Grundbegriffe sowie
einen "Uberblick "uber die biologischen Ph"anomene, die in dieser Arbeit
behandelt werden, liefert Kapitel~\ref{biologie}. Hier werden auch
Ergebnisse einer Analyse experimenteller Daten vorgestellt, welche die
Wechselwirkung verschiedener neuronaler Karten untereinander aufzeigt.

\item In Kapitel \ref{modell} wird das im folgende verwendete,
ph"anomenologische Modell zur Simulation der koordinierten Entwicklung
visueller neuronaler Karten vorgestellt. Wichtige Eigenschaften des Modells
erschlie"sen sich durch eine lineare Stabilit"atsanalyse: So zeigt sich,
da"s die Strukturbildung des Modells kritisch von der
Kooperationsreichweite $\sigma$ abh"angt.  Die Wellenl"angen der
entstehenden Strukturen lassen sich durch geeignete Parameterwahl
vorbestimmen, und k"onnen so an biologische Vorbilder angepasst werden.
Dar"uber hinaus gibt die lineare Stabilit"atsanalyse Auskunft "uber die
Zeitskala der strukturbildenden Dynamik.

\item In Kapitel \ref{main} wird dann ein Szenarium entwickelt, mit dessen
Hilfe sich die in Kapitel \ref{biologie} skizzierten Speziesunterschiede
visueller Reizrepr"asentationen einheitlich erkl"aren lassen.  Die
numerischen Untersuchungen dieses Szenariums anhand des in
Kapitel~\ref{modell} eingef"uhrten Modells ergeben Karten, die sehr gut mit
den beobachteten Karten aus Affen und Katzen "ubereinstimmen.

\item Die Arbeit schlie"st mit einer Diskussion der Ergebnisse in Kapitel
\ref{zusammenfassung}. Unter Voraussetzung der Allgemeing"ultigkeit des
Szenariums werden Vorhersagen "uber die Entstehungsreihenfolge neuronaler
Karten in anderen Spezies gemacht. Alternative Erkl"arungsans"atze werden
diskutiert und Experimente zur Verifikation des Szenariums vorgeschlagen.
\end{itemize}
