\section{Einleitung}
\label{einleitung}
\thispagestyle{plain}

\subsection{Physik im Hirn?}
%... stellt der darwinistischen evolution die sozio-kulturelle evolution zur
%seite, welche dieser teilweise entgegenwirkt.

Verglichen mit anderen Lebewesen auf diesem Planeten nimmt die Spezies
Mensch eine Ausnahmeposition ein. Diese Sonderstellung wird nicht nur
durch morphologische Unterschiede zu unseren nächsten Verwandten wie
z.B. aufrechter Gang und umgestaltetes Gebiß gerechtfertigt. Sie fällt
dem Menschen vielmehr durch die besondere Ausprägung seines Gehirns,
insbesondere des Neocortex zu (siehe Abb.~\ref{hirnmasse}). Das
menschliche Gehirn ist die komplexeste, uns bekannte Struktur im Universum.
Es besteht aus ca. $10^{12}$~Nervenzellen, von denen jede einzelne mit bis
zu tausenden anderen verbunden ist, und ist materielle und funktionale
Grundlage von Wahrnehmen, Empfinden, Erkennen, Lernen, Erinnern, Denken und
Mitteilen; auf seinen Prozessen beruht auch das Bewußtsein. Gleichwohl ist
das Gehirn sich selbst rätselhaft.

Seit ungefähr 100~Jahren gibt es ernstzunehmende Ansätze von
Naturwissenschaftlern, dieses Rätsel zu lösen. Mittlerweile sind zu der
wissenschaftlichen Neugier klare wirtschaftliche und industrielle
Interessen hinzugekommen. So arbeiten in diesem Jahrzehnt --- vom Kongress
der Vereinigten Staaten zur ``Dekade des Hirns'' ausgerufen --- weltweit
Wissenschaftler aus allen Disziplinen daran, die Funktionsweise des
menschlichen Gehirns zu entschlüsseln. Von den Ergebnissen dieser
Forschungen erhofft man sich nicht nur ein besseres Verständnis
neurologischer Krankheitsbilder, wie z.B. der Parkinson'schen Krankheit und
der Multiplen Sklerose. (Dies könnte den Schlüssel zu deren Heilung,
zumindest jedoch Chancen auf bessere Therapien beinhalten.)

Eine besondere Motivation entspringt der Hoffnung, durch die Extraktion
einfacher Regeln und Funktionsprinzipien, die der Informationsverarbeitung
im menschlichen Gehirn zugrunde liegen, die Leistungsfähigkeit
künstlicher informationsverarbeitender Systeme bei bestimmten
Aufgabenstellungen (wie z.B. Mustererkennung, Sprachverständnis,
Spracherkennung, Bewegungskoordination) massiv zu steigern. Auch der sich
durch die ganze Kulturgeschichte ziehende Traum des Menschen von einem
eben--bürtigen, von Menschenhand geschaffenen Wesen, dem \emph{Humanoiden},
wäre dann in greifbare Nähe gerückt \parencite[zum Stand der Dinge siehe z.B.][]{brooks:1990}.

Auf dem Weg zu einem einheitlichen Verständnis von Hirnentwicklung und
\mbox{--funktion} können auch Physiker einen gewinnbringenden Beitrag leisten.
Die moderne Physik bietet eine Palette von Werkzeugen und Methoden zur
Analyse und Beschreibung komplexer Nicht--Gleichgewichts--Systeme. Viele
der in der Theoretischen Physik an einfachen Systemen entwickelten Konzepte
können auf die Hirnforschung übertragen werden, und so auf einer
abstrakten Ebene zum qualitativen Verständnis beobachteter Phänomene
beitragen. Idealerweise werden dabei in der Theorie entwickelte Konzepte
und Modelle im Wechselspiel mit dem Experiment verfeinert. Oft lassen sich
aus solchen Modellen auch Vorhersagen und neue Fragestellungen ableiten.

Diese Ergänzung der experimentellen Hirnforschung um theoretische
Komponenten hat sich in der Vergangenheit bereits für beide Gebiete als
fruchtbar erwiesen: Durch Arbeiten von \textcite{hopfield:1982}
z.B. wurde die statistische Physik als geeigneter Rahmen für die Analyse
der Gedächtnisfunktion sowie der Verallgemeinerung beim Erlernen von
Regeln entdeckt. Die Untersuchung pulsgekoppelter Neurone lieferte neue
Erkenntnisse über die nichtlinearen Synchronisationsmechanismen
oszillierender Systeme \parencite[siehe z.B.][]{ernst95a}. In dieser
Arbeit werden Konzepte der Nichtlinearen Dynamik strukturbildender Systeme
verwendet, um Unterschiede in den visuellen Reizrepräsentationen im Hirn
von Katzen und Affen einheitlich zu erklären. Langfristig dient die
Ausweitung des Kanons der Physik auf biologische Systeme wie z.B. dem
Gehirn wohl dem Erkenntnisfortschritt in beiden Gebieten, der Physik und
der Biologie \parencite{braitenberg:1977}.

\begin{figure}[t]
    \centering
    \epsfig{file=pics/hirngewicht.eps,width=9cm}
    \caption{Man kann die Überentwicklung des menschlichen Hirns bereits
    deutlich am Verhältnis zwischen Gehirn-- und Körpergewicht ablesen. Hier
    dargestellt ist dieses Verhältnis in doppeltlogarithmischer Darstellung
    für 476 Säugetierarten und den Menschen
        \parencite[aus][]{martin:1995}. Das prozentuale Verhältnis ergäbe
        eine Abweichung zugunsten leichtgewichtigerer Arten. Der Wert für den
        Menschen liegt am weitesten oberhalb der Allometriegeraden; unsere Spezies
        hat also relativ zum Körpergewicht ``am meisten'' Hirn.}
    \label{hirnmasse}
\end{figure}

\subsection{Überblick}

In der vorliegenden Arbeit wird ein Modell zur Beschreibung der
aktivitätsabhängigen, koordinierten Entwicklung neuronaler
Reizrepräsentationen (auch: neuronale Karten) untersucht. Basierend auf
diesem Modell wird ein Szenarium vorgeschlagen, das die speziesabhängige
Erscheinung der Reizrepräsentationen aus dem primären visuellen Cortex
von Katzen und Affen einheitlich erklären kann. Die Arbeit gliedert sich
wie folgt:

\begin{itemize}
    \item Eine Einführung in die wichtigsten biologischen Grundbegriffe sowie
    einen Über\-blick über die biologischen Phänomene, die in dieser Arbeit
    behandelt werden, liefert Kapitel~\ref{biologie}. Hier werden auch
    Ergebnisse einer Analyse experimenteller Daten vorgestellt, welche die
    Wechselwirkung verschiedener neuronaler Karten untereinander aufzeigt.

    \item In Kapitel~\ref{modell} wird das im folgende verwendete,
    phänomenologische Modell zur Simulation der koordinierten Entwicklung
    visueller neuronaler Karten vorgestellt. Wichtige Eigenschaften des Modells
    erschließen sich durch eine lineare Stabilitätsanalyse: So zeigt sich,
    daß die Strukturbildung des Modells kritisch von der
    Kooperationsreichweite $\sigma$ abhängt. Die Wellenlängen der
    entstehenden Strukturen lassen sich durch geeignete Parameterwahl
    vorbestimmen, und können so an biologische Vorbilder angepasst werden.
    Darüber hinaus gibt die lineare Stabilitätsanalyse Auskunft über die
    Zeitskala der strukturbildenden Dynamik.

    \item In Kapitel~\ref{main} wird dann ein Szenarium entwickelt, mit dessen
    Hilfe sich die in Kapitel~\ref{biologie} skizzierten Speziesunterschiede
    visueller Reizrepräsentationen einheitlich erklären lassen. Die
    numerischen Untersuchungen dieses Szenariums anhand des in
    Kapitel~\ref{modell} eingeführten Modells ergeben Karten, die sehr gut mit
    den beobachteten Karten aus Affen und Katzen übereinstimmen.

    \item Die Arbeit schließt mit einer Diskussion der Ergebnisse in Kapitel
    ~\ref{zusammenfassung}. Unter Voraussetzung der Allgemeingültigkeit des
    Szenariums werden Vorhersagen über die Entstehungsreihenfolge neuronaler
    Karten in anderen Spezies gemacht. Alternative Erklärungsansätze werden
    diskutiert und Experimente zur Verifikation des Szenariums vorgeschlagen.
\end{itemize}
