\section{Numerische Integration der Dynamik}
\label{anhang2}
\thispagestyle{plain}

In allen Simulationen zum Szenarium wurde die Dynamik~\eqref{endyn}
ausgehend vom Anfangszustand $\mathbf{R}_0(\mathbf{x}) = (x,\, y,\, 0,\,
0,\, 0) $ auf einem diskreten Gitter aus $40\times 40$ bzw. $64\times 64$
Neuronen mit periodischen Randbedingungen integriert. Die Varianzen
$\left<S_i^2\right>$ der Reizverteilungen wurden dabei so gewählt, daß
die enstehenden Muster vorgegebene Wellenlängen aufweisen
(vgl. Abschn.~\ref{numerg}). Dazu wurde die Gleichung

\begin{equation*}
    \Lambda_i = \frac{2 \pi}{\sqrt{\ln\left(\left<S_i^2\right>/\eta\right)}}\;\sigma_i
\end{equation*}

\noindent numerisch invertiert. Die Gültigkeit dieser Gleichung
erfordert, daß $\sigma$ während der Simulation monoton fällt und zu
Beginn größer als das größte kritische~$\sigma^\ast$ ist. Der
Zeitverlauf von $\sigma(t)$ wurde in der Regel als linearer Abfall von den
Werten $\sigma_{\mbox{\tiny Start}}=1.1*\sigma_i^\ast$ und
$\sigma_{\mbox{\tiny Ende}}=0.9*\sigma_j^\ast$, mit
$\sigma_i>\sigma_j$ angesetzt (vgl. Abb.~\ref{zeitentwicklung}).

Die numerische Integration der Dynamik~\eqref{endyn} erfolgte dann in einem
expliziten Einschritt--Verfahren. Für den Integralterm der Dynamik wurde
ein Adams--Bashford Schritt verwendet, der diffusionsartige Anteil der
Dynamik anschließend in Fourierdarstellung exakt integriert. Dieses
Verfahren gewährleistet die numerische Stabilität des linearen Terms.
Der Integralterm wurde als Mittelwert über eine Menge von
Pseudo--Zufallsstimuli genähert. Die Anzahl der Stimuli wurde dabei so
gewählt, daß in jedes Volumenelement der linearen Ausdehnung $2\sigma$
mindestens 50 Reize fallen. Typischerweise erforderte dies ca.~20000
Stimuli pro Integrationsschritt.

Die Wahl des Integrationszeitschrittes $dt$ orientierte sich unter
Zuhilfenahme der Analytik an drei Kriterien: $dt$ sollte klein auf der
Zeitskala der linearen Instabilität, und nicht zu groß für den
Integralterm sein (dieser zeigt eine numerische Instabilität bei großen
$dt$). Außerdem sollte $dt$ klein sein gegenüber der Zerfallszeit des
Musters unter der Einwirkung des Laplace--Schritts. Diese Randbedingungen
der Wahl von $dt$ wurden durch

\begin{equation*}
    dt = \text{MIN}\left(\frac{\tau_{\mbox{\tiny
        Inst}}}{5},\frac{\tau_{\scriptscriptstyle \Delta}}{20}\right)
\end{equation*}

\noindent erfüllt, wobei $\tau_{\mbox{\tiny
    Inst}}=\frac{1}{\lambda(k_{\text{max}})}$ und $\tau_{\scriptscriptstyle
    \Delta}=\eta*k_{\text{max}}^2$. Die Zeitentwicklung einer durch die
Instabilität entstehenden Struktur wurde in einem Zeitraum von bis zu
$200\tau$ verfolgt. Eine typische Simulation nahm auf modernsten
Workstations mehrere Wochen in Anspruch.
