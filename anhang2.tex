\section{Numerische Integration der Dynamik}
\label{anhang2}
\thispagestyle{plain}

In allen Simulationen zum Szenarium wurde die Dynamik~\eqref{endyn}
ausgehend vom Anfangszustand $\mathbf{R}_0(\mathbf{x}) = (x,\, y,\, 0,\,
0,\, 0) $ auf einem diskreten Gitter aus $40\times 40$ bzw. $64\times 64$
Neuronen mit periodischen Randbedingungen integriert.  Die Varianzen
$\left<S_i^2\right>$ der Reizverteilungen wurden dabei so gew"ahlt, da"s
die enstehenden Muster vorgegebene Wellenl"angen aufweisen
(vgl. Abschn.~\ref{numerg}). Dazu wurde die Gleichung

\begin{equation*}
\Lambda_i = \frac{2 \pi}{\sqrt{\ln\left(\left<S_i^2\right>/\eta\right)}}\;\sigma_i
\end{equation*}

\noindent numerisch invertiert.  Die G"ultigkeit dieser Gleichung
erfordert, da"s $\sigma$ w"ahrend der Simulation monoton f"allt und zu
Beginn gr"o"ser als das gr"o"ste kritische~$\sigma^\ast$ ist.  Der
Zeitverlauf von $\sigma(t)$ wurde in der Regel als linearer Abfall von den
Werten $\sigma_{\mbox{\tiny Start}}=1.1*\sigma_i^\ast$ und
$\sigma_{\mbox{\tiny Ende}}=0.9*\sigma_j^\ast$, mit
$\sigma_i>\sigma_j$ angesetzt (vgl. Abb.~\ref{zeitentwicklung}).

Die numerische Integration der Dynamik~\eqref{endyn} erfolgte dann in einem
expliziten Einschritt--Verfahren. F"ur den Integralterm der Dynamik wurde
ein Adams--Bashford Schritt verwendet, der diffusionsartige Anteil der
Dynamik anschlie"send in Fourierdarstellung exakt integriert.  Dieses
Verfahren gew"ahrleistet die numerische Stabilit"at des linearen Terms.
Der Integralterm wurde als Mittelwert "uber eine Menge von
Pseudo--Zufallsstimuli gen"ahert. Die Anzahl der Stimuli wurde dabei so
gew"ahlt, da"s in jedes Volumenelement der linearen Ausdehnung $2\sigma$
mindestens 50 Reize fallen. Typischerweise erforderte dies ca.~20000
Stimuli pro Integrationsschritt.

Die Wahl des Integrationszeitschrittes $dt$ orientierte sich unter
Zuhilfenahme der Analytik an drei Kriterien: $dt$ sollte klein auf der
Zeitskala der linearen Instabilit"at, und nicht zu gro"s f"ur den
Integralterm sein (dieser zeigt eine numerische Instabilit"at bei gro"sen
$dt$). Au"serdem sollte $dt$ klein sein gegen"uber der Zerfallszeit des
Musters unter der Einwirkung des Laplace--Schritts.  Diese Randbedingungen
der Wahl von $dt$ wurden durch

\begin{equation*}
dt = \text{MIN}\left(\frac{\tau_{\mbox{\tiny
Inst}}}{5},\frac{\tau_{\scriptscriptstyle \Delta}}{20}\right)
\end{equation*}

\noindent erf"ullt, wobei $\tau_{\mbox{\tiny
Inst}}=\frac{1}{\lambda(k_{\text{max}})}$ und $\tau_{\scriptscriptstyle
\Delta}=\eta*k_{\text{max}}^2$.  Die Zeitentwicklung einer durch die
Instabilit"at entstehenden Struktur wurde in einem Zeitraum von bis zu
$200\tau$ verfolgt. Eine typische Simulation nahm auf modernsten
Workstations mehrere Wochen in Anspruch.
